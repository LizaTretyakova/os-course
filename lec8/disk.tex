\begin{frame}
\frametitle{Устройства харнения}
\begin{itemize}
  \item<1-> Есть довольно много различных устройств хранения:
    \begin{itemize}
      \item магнитные ленты (и да они до сих пор используются);
      \item оптические диски;
    \end{itemize}
  \item<2-> Нас будут интересовать два довольно конкретных вида:
    \begin{itemize}
      \item механические магнитные диски (HDD);
      \item flash-based диски (SSD);
    \end{itemize}
\end{itemize}
\end{frame}

\begin{frame}
\frametitle{Адресация блоков диска}
\begin{itemize}
  \item<1-> Обмен данных с диском происходит блоками фиксированного размера
    \begin{itemize}
      \item поэтому устройства страницы обычно называют блочными устройствами (выполняют запросы на передачу N блоков)
      \item типичный размер блока 512 байт (возможно увеличится в ближайшее время)
      \item блоки нумеруются последовательно - это называется LBA;
    \end{itemize}
  \item<2-> Ранее для HDD использовалась другая адресация - CHS
    \begin{itemize}
      \item CHS отражает физическое устройство HDD
      \item С - cylinder
      \item H - head
      \item S - sector
    \end{itemize}
\end{itemize}
\end{frame}

\begin{frame}
\frametitle{Время доступа к диску}
\begin{itemize}
  \item<1-> Время выполнения запроса к HDD состоит из нескольких параметров:
    \begin{itemize}
      \item время позиционирования головки диска (2-4 мс);
      \item время ротации - ждем пока нужный сектор окажется под головкой (2-8 мс);
      \item время передачи данных (~30 мкс);
    \end{itemize}
  \item<2-> Последовательный доступ быстрее чем случайный:
    \begin{itemize}
      \item при последовательном доступе происходит меньше позиционирований;
      \item для SSD дисков это тоже справедливо, но в силу других причин;
    \end{itemize}
\end{itemize}
\end{frame}

\begin{frame}
\frametitle{Планирование IO}
\begin{itemize}
  \item<1-> Запросы к диску должны быть большими и редкими
    \begin{itemize}
      \item несколько процессов обращаются к диску независимо;
      \item приложения выдают запросы по частям (например, блоками по 4 KB);
      \item запросы на чтение и запись отличаются;
    \end{itemize}
  \item<2-> Задачу накапливания/слияния запросов к диску выполняет планировщик IO
    \begin{itemize}
      \item планирование должно быть честным;
      \item планировщик не поможет с случайными запросами;
      \item иногда параллельность важнее последовательности;
    \end{itemize}
\end{itemize}
\end{frame}
