\begin{frame}
\frametitle{Потребление памяти процессами}
\begin{itemize}
  \item<1-> Большинство процессов в системе потребляет небольшое количество
            ресурсов
    \begin{itemize}
      \item например, у меня в системе в среднем около 250 процессов;
      \item все вместе они используют около 3 GB памяти из 8 GB;
      \item больше половины занятой памяти используется 10 \% процессов;
    \end{itemize}
  \item<2-> Свободная память не приносит пользы:
    \begin{itemize}
      \item без разницы 100 GB памяти или 200 GB у вас в системе, если вы
            используете только 3 GB;
      \item полезно держать небольшой запас память на всякий случай - но нам не
            нужно для этого несколько GB
    \end{itemize}
\end{itemize}
\end{frame}

\begin{frame}
\frametitle{Buffer Cache}
\begin{itemize}
  \item<1-> Доступ к диску медленный:
    \begin{itemize}
      \item доступ к диску как правило обладает локальностью:
        \begin{itemize}
          \item временная локальность - обращение к данным, к которым мы уже
                обращались недавно;
          \item пространственная локальность - обращение к данным, которые
                находятся на диске рядом с данными, к которым мы обращались;
        \end{itemize}
    \end{itemize}
  \item<2-> У нас есть запас в несколько GB свободной памяти:
    \begin{itemize}
      \item кеш в несколько GB при наличии локальности может дать большой
            прирост в производительности;
      \item мы всегда можем уменьшить кеш, если нам не хватает памяти;
    \end{itemize}
\end{itemize}
\end{frame}

\begin{frame}
\frametitle{Стратегии замещения и записи}
\begin{itemize}
  \item<1-> Размер кеша ограничен
    \begin{itemize}
      \item когда-нибудь он заполнится и придется решать кого выкинуть из кеша;
      \item выкинем сектор, к которому дольше всего не было обращений - LRU;
    \end{itemize}
  \item<2-> Buffer Cache позволяет ускорять запись
    \begin{itemize}
      \item мы записываем данные в кеш - дальше кеш разберется сам;
      \item нужно решить когда записывать данные на диск
        \begin{itemize}
          \item мы можем начать запись на диск, как только мы добавили данные в
                кеш - write-through;
          \item мы можем отложить запись на какое-то время - write-back;
        \end{itemize}
    \end{itemize}
\end{itemize}
\end{frame}


